\documentclass[11pt,a4paper]{article}
\usepackage[spanish,es-nodecimaldot]{babel}	% Utilizar español
\usepackage[utf8]{inputenc}					% Caracteres UTF-8
\usepackage{graphicx}						% Imagenes
\usepackage[hidelinks]{hyperref}			% Poner enlaces sin marcarlos en rojo
\usepackage{fancyhdr}						% Modificar encabezados y pies de pagina
\usepackage{float}							% Insertar figuras
\usepackage[textwidth=390pt]{geometry}		% Anchura de la pagina
\usepackage[nottoc]{tocbibind}				% Referencias (no incluir num pagina indice en Indice)
\usepackage{enumitem}						% Permitir enumerate con distintos simbolos
\usepackage[T1]{fontenc}					% Usar textsc en sections
\usepackage{amsmath}				% Símbolos matemáticos
\usepackage{listings}
\usepackage{color}

 
\definecolor{codegreen}{rgb}{0,0.6,0}
\definecolor{codegray}{rgb}{0.5,0.5,0.5}
\definecolor{codepurple}{rgb}{0.58,0,0.82}
\definecolor{backcolour}{rgb}{0.95,0.95,0.95}
 
\lstdefinestyle{mystyle}{
    backgroundcolor=\color{backcolour},   
    commentstyle=\color{codegreen},
    keywordstyle=\color{magenta},
    numberstyle=\tiny\color{codegray},
    stringstyle=\color{codepurple},
    basicstyle=\footnotesize,
    breakatwhitespace=false,         
    breaklines=true,                 
    captionpos=b,                    
    keepspaces=true,                 
    numbers=left,                    
    numbersep=5pt,                  
    showspaces=false,                
    showstringspaces=false,
    showtabs=false,                  
    tabsize=2
}
 
\lstset{style=mystyle, language=C++}

% Comando para poner el nombre de la asignatura
\newcommand{\asignatura}{Simulación de Sistemas}
\newcommand{\autor}{José María Sánchez Guerrero}
\newcommand{\titulo}{Ejercicio de MonteCarlo}
\newcommand{\subtitulo}{Fábrica de chocolate}

% Configuracion de encabezados y pies de pagina
\pagestyle{fancy}
\lhead{\autor{}}
\rhead{\asignatura{}}
\lfoot{Grado en Ingeniería Informática}
\cfoot{}
\rfoot{\thepage}
\renewcommand{\headrulewidth}{0.4pt}		% Linea cabeza de pagina
\renewcommand{\footrulewidth}{0.4pt}		% Linea pie de pagina

\begin{document}
\pagenumbering{gobble}

% Pagina de titulo
\begin{titlepage}

\begin{minipage}{\textwidth}

\centering

\includegraphics[scale=0.5]{img/ugr.png}\\

\textsc{\Large \asignatura{}\\[0.2cm]}
\textsc{GRADO EN INGENIERÍA INFORMÁTICA}\\[1cm]

\noindent\rule[-1ex]{\textwidth}{1pt}\\[1.5ex]
\textsc{{\Huge \titulo\\[0.5ex]}}
\textsc{{\Large \subtitulo\\}}
\noindent\rule[-1ex]{\textwidth}{2pt}\\[3.5ex]

\end{minipage}

\vspace{0.5cm}

\begin{minipage}{\textwidth}

\centering

\textbf{Autor}\\ {\autor{}}\\[2.5ex]
\textbf{Rama}\\ {Computación y Sistemas Inteligentes}\\[2.5ex]
\vspace{0.3cm}

\includegraphics[scale=0.3]{img/etsiit.jpeg}

\vspace{0.7cm}
\textsc{Escuela Técnica Superior de Ingenierías Informática y de Telecomunicación}\\
\vspace{1cm}
\textsc{Curso 2019-2020}
\end{minipage}
\end{titlepage}

\pagenumbering{arabic}
\tableofcontents
\thispagestyle{empty}				% No usar estilo en la pagina de indice

\newpage

\setlength{\parskip}{1em}


\section{Introducción}

Una fábrica de chocolate recibe todos los años para diciembre un pedido de huevos de Pascua. Por razones estacionales, resulta más barato comprar el
chocolate necesario durante el mes de \textbf{agosto}, asi que la empresa compra una gran cantidad de chocolate este mes, y si es necesario comprar más,
se realiza otro \textbf{pedido adicional} para satisfacer de forma exacta toda la demanda. Por otro lado, si el chocolate comprado en agosto sobra, será
donado a comedores de escuelas.

También tenemos los siguientes datos sobre precios y cantidades:
\begin{itemize}
	\item Cada huevo de pascua emplea 250 gramos de chocolate.
	\item El precio del chocolate en agosto es de 1 euro por kilo.
	\item El precio del chocolate en diciembre es de 1.5 euros por kilo.
	\item El precio de venta de los huevos de pascua es de 0.60 euros la unidad.
\end{itemize}

La demanda de huevos al año sigue una distribucion triangular, con valor más probable es \textbf{c = 2600 unidades}, el menor valor es \textbf{a = 2000
unidades} y mayor valor es \textbf{b = 3000 unidades}. Su función de densidad es la siguiente:
\begin{equation*}
	f(x)=
	\left\{\begin{matrix}
	\frac{2(x-a)}{(b-a)(c-a)} & a\leq x \leq c\\ 
	\frac{2(b-x)}{(b-a)(b-c)} & c\leq x \leq b 
	\end{matrix}\right.
\end{equation*}

Con esta información tendremos que construir un modelo de simulación 



\section{Implementación}


La implementación de este generador triangular ya se nos proporcionará en el enunciado. Primero necesitamos un primer generador de números aleatorios:
\begin{lstlisting}
float uniforme(){
	float u;
	u = (float) random();
	u = u/(float)(RAND_MAX+1.0);
	return u;
}
\end{lstlisting}

Y después lo utilizamos en la función de distribución comentada anteriormente:
\begin{lstlisting}
int generademanda(){
	float u, x;
	u = uniforme();
	
	if (u < (c-a)/(b-a))
		x = a+sqrt((b-a)*(c-a)*u);
	else
		x = b-sqrt((b-a)*(b-c)*(1-u));
	
	return (int) x; // Se convierte a entero porque es la demanda
                  // de huevos de pascua
}
\end{lstlisting}







\end{document}



